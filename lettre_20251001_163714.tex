\documentclass[11pt]{article}
\usepackage[a4paper,margin=1cm]{geometry}
\usepackage{helvet}
\renewcommand{\familydefault}{\sfdefault}
\usepackage[T1]{fontenc}
\usepackage[utf8]{inputenc}
\usepackage{array}
\usepackage{xcolor}
\usepackage[hidelinks]{hyperref}
\usepackage{enumitem}
\usepackage{pifont}
\usepackage{amssymb}

\usepackage{graphicx}
\usepackage[absolute,overlay]{textpos}  % pour position libre de l'image
\usepackage{luacode}

\setlength{\parindent}{0pt}

% === STYLE DES SECTIONS ===
\usepackage{titlesec}
\titleformat{\section}
  {\large\bfseries}
  {}
  {0pt}
  {}[\vspace{-0.2cm}\rule{\textwidth}{2pt}]
\titlespacing*{\section}{0pt}{15pt}{0pt}

\begin{document}

% ==== Télécharger l'image depuis GitHub RAW avec Lua ====
\begin{luacode*}
local url = "https://raw.githubusercontent.com/azizbelhadjsayar/azizbelhadjsayar.github.io/main/images/profilepic.jpg"
local http = require("socket.http")
local body, code = http.request(url)

if code == 200 then
  local f = io.open("tempimg.jpg","wb")
  f:write(body)
  f:close()
else
  tex.print("Impossible de télécharger l'image : "..code)
end
\end{luacode*}

% ==== Afficher l'image devant le texte à la position souhaitée ====
\begin{textblock*}{3cm}(14cm,0.5cm) % largeur x hauteur, position (x,y) en cm
\includegraphics[width=3cm]{tempimg.jpg}
\end{textblock*}

% ==== HEADER ====
{\Huge\bfseries AZIZ BELHADJ SAYAR}\\[2pt]
{\large Computer Science Student}\\[0pt]

% ==== Coordonnées ====
\renewcommand{\arraystretch}{0.9}
\setlength{\tabcolsep}{6pt}
\noindent
\begin{tabular*}{0.8\textwidth}{@{\extracolsep{\fill}} l l l}
    \textcolor{blue}{\ding{37}} +33 6 12 34 56 78 & 
    \textcolor{red}{\ding{43}} Paris, France & 
    \textcolor{teal}{\href{https://github.com/azizbelhadj}{\ding{202} github.com/azizbelhadj}} \\
    \textcolor{orange}{\Letter} \href{mailto:aziz.belhadj@example.com}{aziz.belhadj@example.com} & 
    \textcolor{cyan}{\ding{70}} \href{https://linkedin.com/in/azizbelhadj}{linkedin.com/in/azizbelhadj} & 
    \textcolor{purple}{\ding{43}} \href{https://portfolio.aziz.dev}{portfolio.aziz.dev} \\
\end{tabular*}

\vspace{0cm}

% ==== Sections du CV ====
\section*{Profil}
Étudiant passionné par la Business Intelligence et le développement logiciel, 
actuellement en Master 1 Software Engineering for the Web à l’Université Paris-Saclay.

\section*{Formation}
\noindent
\begin{tabular*}{\textwidth}{@{\extracolsep{\fill}} l r}
\textbf{Université Paris-Saclay} & \textbf{Paris, France} \\
\textbf{Master 1 Software Engineering for the Web} & \textbf{2024 -- 2025} \\
\end{tabular*}
\begin{itemize}[leftmargin=*,itemsep=1pt,topsep=0pt,parsep=0pt,label=\textcolor{blue}{$\rightarrow$}]
    \item Modules principaux : Software Engineering, Web Development, Databases
    \item Projet de fin d'études en développement d'application Web avec Python/Angular
\end{itemize}

\vspace{0.2cm}

\noindent
\begin{tabular*}{\textwidth}{@{\extracolsep{\fill}} l r}
\textbf{IHEC Carthage} & \textbf{Tunis, Tunisie} \\
\textbf{Licence en Informatique de Gestion (Business Intelligence)} & \textbf{2021--2024} \\
\end{tabular*}
\begin{itemize}[leftmargin=*,itemsep=1pt,topsep=0pt,parsep=0pt,label=\textcolor{red}{$\rightarrow$}]
    \item Modules principaux : Business Intelligence, SQL, Data Analysis
    \item Projets : analyse de données réelles, dashboard interactif avec Python
\end{itemize}

\section*{Expérience Professionnelle}
\noindent
\begin{tabular*}{\textwidth}{@{\extracolsep{\fill}} l r}
\textbf{Tech Solutions} & \textbf{Paris, France} \\
\textbf{Développeur Web Junior} & \textbf{2023--2024} \\
\end{tabular*}
\begin{itemize}[leftmargin=*,itemsep=1pt,topsep=0pt,parsep=0pt,label=\textcolor{green}{$\rightarrow$}]
    \item Développement d’une application web interne en Python/Angular
    \item Maintenance et optimisation de bases de données SQL
    \item Collaboration avec l’équipe BI pour l’analyse de données
\end{itemize}

\section*{Projets}
\noindent
\begin{tabular*}{\textwidth}{@{\extracolsep{\fill}} l r}
\textbf{Dashboard interactif Python} & \textbf{2024} \\
\end{tabular*}
\begin{itemize}[leftmargin=*,itemsep=1pt,topsep=0pt,parsep=0pt,label=\textcolor{orange}{$\rightarrow$}]
    \item Création d’un dashboard interactif pour l’analyse de données réelles
\end{itemize}

\end{document}
